% @author Hani Alshikh
%
\chapter{Installation and Configuration}

Installing and configuring the \gls{gl:ac} and \gls{gl:ab} spans multiple resources and requires intricate yet clear and logical orchestration.

Such processes tend to be automated in production environments to insure auditable and reversible actions through \gls{gl:vc} systems, speed up development, reduce human errors, and benefits from \gls{ac:ci}/\gls{ac:cd} practices.

Utilizing \gls{gl:docker}, \gls{ac:k8s}, \gls{gl:helm} and various backing technologies while keeping \gls{gl:12faktor}~\citep{TheTwelv47} rules in check (\hyperref[oc:ac02]{OC02}) the entire installation and configuration process also known as a deployment can be trimmed down to just a few steps.

\subsection{Deployment View}

The following is a general overview of the main resources and services to successfully deploy and use the \gls{gl:ac} and \gls{gl:ab} 

\includenamedsvg{\includesvg[width=\textwidth,pretex=\relscale{0.1}]}{diagrams/deployment_view.drawio}{Deployment view diagram}

Running \gls{gl:monogui} and the \gls{gl:grpcweb} proxy on separate nodes is not necessary but supported to allow for flexible deployment usecases.

\section{Test Run}

For the local installation all resources and services will be deployed on the same \gls{ac:k8s} node. Since \gls{ac:m8} already uses \gls{gl:emissary} it will be used as the \gls{gl:grpcweb} proxy. \Gls{gl:dex} will be used as the \gls{ac:idp}.

To deploy \gls{ac:m8} and \gls{gl:monogui} along all other resources to a local cluster and experiment with the \gls{gl:ac} and \gls{gl:ab} a script is provided, that will:

\begin{enumerate}
  \item take care of the preparation work and installing the automated~(\ref{hdr:auto}) tools
  \item create local \gls{ac:k8s} cluster using \gls{gl:kind}
  \item setup and deploy all needed resources
\end{enumerate}

All while making sure to stay \gls{gl:idpt} and localized.

\header{Prerequisite}

- A supported system from the following:

\begin{itemize}
  \item MacOS
  \item Linux (tested on Ubuntu)
  \item Windows (WSL should work but not tested)
\end{itemize}

For a sandboxed run \gls{gl:vb} and Ubuntu can be used. Providers like \gls{gl:vmimg} offer a ready to boot images, that runs with zero installation and configuration efforts

- Standard userspace utilities like \lstinline[columns=fixed]{bash}, \lstinline[columns=fixed]{curl}, \lstinline[columns=fixed]{tar}, etc... should be available natively but can always be changed to alternatives by setting the corresponding environment variable. For example replacing \sh{curl} with \sh{wget} will be as follows:

\begin{lstlisting}[language=bash, caption={Deploy with custom command}, label={sh:dwcc}]
    $ CURL=wget make deploy
\end{lstlisting}

- The following tools:

\begin{table}[H]
  \begin{center}
    \label{tab:irt}
    \def\arraystretch{1.5}
    \begin{tabularx}{\linewidth}{|l X |}
      \hline
      \rowcolor{gray!20}
      \textbf{Tool} & \textbf{Reson}\\
      \gls{gl:docker} & to create isolated environments\\
      \gls{gl:make} & to run deployment scripts\\
      \hline
    \end{tabularx}
    \caption{Installation required tools}
  \end{center}
\end{table}

\header{Automated}\label{hdr:auto}

The following will be downloaded and configured locally. If any should fail please download and install manually then follow the same instruction as described in the shell example~\ref{sh:dwcc}.

\begin{table}[H]
  \begin{center}
    \label{tab:iat}
    \def\arraystretch{1.5}
    \begin{tabularx}{\linewidth}{|l X |}
      \hline
      \rowcolor{gray!20}
      \textbf{Tool} & \textbf{Reson}\\
      \gls{gl:kubectl} & to manage \gls{ac:k8s} \\
      \gls{gl:kind} & to create a local \gls{ac:k8s} cluster\\
      \gls{gl:helm} & to generate components resources and install required \glspl{ac:crd}\\
      \gls{gl:step} & to generate \gls{ac:m8} \glspl{ac:pki} trust-anchor\\
      \hline
    \end{tabularx}
    \caption{Installation automated tools}
  \end{center}
\end{table}

\subsection{Configure}

Depending on the version on hand open the provided CD-Drive or navigate to \url{https://thesis.alshikh.de} and clone the repository.

As specified by \gls{gl:12faktor}~\citep{TheTwelv47} methodology (\hyperref[oc:ac02]{OC02}) all configuration are environment specifc and can be configured locally or using \gls{ac:k8s} resource definiations by setting the the corresponding enviorment variable.

All variables can be found under the directory \sh{deploy/setup} and set with defaults for ease of use.

\subsection{Install}

Open the directory \sh{deploy} and follow the following instruction\footnote{If for some reason a command should fail the system and commands are ensured to be \gls{gl:idpt}. Just retry.}.

\begin{enumerate}
  \item Create a local cluster and deploy all resources:

  \begin{lstlisting}[language=bash, caption={Deploy all resources to a local cluster}, label={sh:dartlc}]
    
    $ make deploy
    # you can also run the following 
    # to monitor the state of the resources
    $ make kind-watch
  \end{lstlisting}

  \item Trust \gls{ac:m8} \acrshort{ac:ca}: \sh{tmp/ca.crt}, otherwise the browser will block communication with \gls{ac:m8} \gls{ac:api}

  \begin{lstlisting}[language=bash, caption={Trust \gls{ac:m8} domain certificate}, label={sh:tmdc}]
    
    # MacOS
    $ make turst-m8-ca
    # on ubunte please add the \gls{ac:ca} 
    # to the browser specific store
  \end{lstlisting}

  \item Add the following to your hosts file:

  \begin{lstlisting}[language=bash, caption={Update hosts file}, label={sh:uhf}]
    
    # Linux & MacOS: /etc/hosts
    # Windows: c:\Windows\System32\Drivers\etc\hosts
    127.0.0.1 api.monoskope.dev
    127.0.0.1 dex
  \end{lstlisting}

  \pagebreak

  \item Create port-forwards to route local request to backing services in the cluster: (Make sure \sh{8443}, \sh{5556} and \sh{3000} are not in use)

  \begin{lstlisting}[language=bash, caption={Create port-forwards to route local request}, label={sh:cpf}]
    
    $ make port-forward
  \end{lstlisting}

  \item navigate to \href{http://localhost:3000}{http://localhost:3000} and sign in using the following credintials:
  
  \begin{itemize}
    \item \sh{username:} admin@monoskope.dev
    \item \sh{password:} password
  \end{itemize}

  \item Populate the \gls{gl:est} with some data for a better \gls{ac:ux}

  \begin{lstlisting}[language=bash, caption={Populate the \gls{gl:est} with some mock}, label={sh:pemd}]
    
    $ make mock-data
  \end{lstlisting}
\end{enumerate}
