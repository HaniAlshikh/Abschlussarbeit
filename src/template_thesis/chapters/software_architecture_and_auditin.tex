% @author Hani Alshikh
%
\chapter{Software Architecture and Auditing}\label{chap:sadt}

Software architecture is the structure, or set of structures, which comprises software elements, the externally visible properties of those elements, and the relationships among them \citep{SAIP}. 

This structure is an artifact from a software development process and is represented by a document composed by one or more models, which represent different perspectives about how the system will be structured, and information sets, that facilitate the understanding of the proposed computational solution. 

The software architecture is defined based on the software requirements. Among the different types of requirements, the quality requirements are the most important for the specification of an architecture since it exerts considerable influence over it structure \citep{SAIP}.

Having a system that is designed and built from the ground up to offer the highest inclusion of auditing standards and capabilities starts by choosing the right architectural components. Auditing functionalities is often added as an afterthought, resulting in an inherent risk of incompleteness. 

Avoiding such risk and insuring auditability by design can be done in many different ways. Ultimately, the choice of architecture patterns for auditing will depend on the specific needs of the system and the requirements of the audit process.

As discussed in section \ref{sec:adtlog} audit logging is one of the simplest way to have an auditable system, that satisfies regulators specifications.

\pagebreak

\section{Implementing audit logging}\label{sec:adtimpl}

The glory of the Audit Log pattern is its simplicity. Comparing Audit Log to other patterns such as Temporal Property~\citep{TemporalProperty} and Temporal Object~\citep{TemporalObject} shows, that these alternatives add a lot of complexity to an object model, although these are both often better at hiding that complexity than using Effectivity~\citep{Effectiv17} everywhere~\citep{AuditLog}.

As described in section \ref{sec:adtlog} simply writing log entries into a file is not enough. At least when ensuring business provenance [\ref{sec:bsprov}] is a requirement. Provenance data will make it possible to replay history reliably and accurately and to predict problems, thereby improving business processes.

Using Audit Log in some parts of the system and other patterns elsewhere is common and sometimes necessary. For example Audit Log can be used in combination with \gls{gl:es} for state changes and a different pattern for read operations.

\citep{richardson2018microservices} describes three main ways to implement audit logging:

\begin{enumerate}
  \item Add audit logging code to the business logic.
  \item Use \gls{ac:aop}.
  \item Use \gls{gl:es}.
\end{enumerate}

\subsection{Audit Logging Code in Business Logic}

The first and most straightforward option is to sprinkle audit logging code throughout the service’s business logic. Each service method, for example, can create an audit log entry and save it in the database. 

The drawback with this approach is that it intertwines audi logging code and business logic, which reduces maintainability.
The other drawback is that it is potentially error prone, because it relies on the developer writing audit logging code and insuring compliance.

Logging in general is mostly associated with debugging and has no direct relation to the system state. Logging style and verbosity might negatively effect the integrity and completeness of the audit log.

\subsection{\acrlong{ac:aop}}

The second option is to use \gls{ac:aop}. Frameworks like spring offer \gls{ac:aop} support, which automatically intercepts each service method call and persists an audit log entry.

This is a much more reliable approach, because it automatically records every service method invocation, which insures a complete audit log of both reading and writing operations.

The main drawback of using \gls{ac:aop} is that it only has access to the method name and its arguments, thus it might be challenging to determine the business object being acted upon and generate a business-oriented audit log entry.

\subsection{Event Sourcing}\label{sec:saes}

The third option is to implement the business logic with auditability as a first class property by utilising \gls{gl:es} and its complementary patterns as discussed in chapter \ref{chap:es}. 

\gls{gl:es} offer an audit log by design for all state changing operations. The main limitation when using \gls{gl:es} is the missing records for read operations.

If logging read operations is a requirement, using one of the other options is a must. Patterns like \gls{ac:cqrs} help encapsulate the different implementations and unify the auditing \gls{ac:api}

\section{What to Consider}

\subsection{Traditional Persistence}

When it comes to auditing a limitation of the traditional persistence methods is that, they only store the current state of a business object. Once the object has been updated, its previous state is lost. 

Such historical records are the base for services like Asana or Jira. Keeping track of such changes, while insuring association to the correct object and actor is a challenge, that comes with its added complexity.

Simply logging different kind of operations is not enough. If tracking objects history and insuring association between each log entry, the object involved and the actor initiating the operation is a requirement choosing the right pattern makes all the difference.

The challenge of implementing auditing as an added feature like the case with the first two options discussed in section \ref{sec:adtimpl} is that, besides being a time-consuming chore, the audit logging code and the business logic can diverge, resulting in all different kind of bugs.

\subsection{Auditing 2.0}\label{sec:adt2sa}

Performing and supporting IT audits and managing IT audit programs are time-, effort-, and personnel-intensive activities~\citep{GANTZ20141-T}. In an age of cost-consciousness and competition for resources, it is reasonable to keep \gls{gl:adt2} in mind when implementing Audit Log. Having an audit log, that can be utilised by the different process mining techniques as described in section \ref{sec:psmin} enables new forms of auditing. Rather than sampling a small set of cases, the whole process and all of its instances can be considered. Moreover, this can be done continuously.

Auditors can utilise process mining techniques to address multiple use-cases/processes. By implementing patterns like \gls{gl:es} organizations can use data analytics and machine learning techniques to analyze the event data and identify patterns and trends that may indicate potential risks or issues. By continuously analyzing the event data, organizations can proactively identify and address potential problems as they arise, rather than waiting for them to be discovered during a periodic audit.