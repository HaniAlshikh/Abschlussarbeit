% @author Hani Alshikh
%
\chapter{Audit Browser}

Wenn mehrere Teil-Kapitel zu strukturieren sind: Schreiben Sie zu jedem Teil-Kapitel eine Ein- leitung ("Hier wird die folgende Fragestellung untersucht...") und eine Ausleitung ("Hiermit ist erreicht: ... Die folgenden Probleme sind aber noch offen:...").

Process Discovery \cite{5427384}
By analyzing frequent patterns, process mining techniques can extract from event logs models that describe the processes at hand. For example, the Alpha process mining algorithm can automatically extract a Petri net that concisely models behavior in the event log. This gives the auditor an unbiased view of what has actually happened.

Conformance Checking
An auditor can use an a priori process model to check if reality, as recorded in the event log, conforms to the model and vice versa. For example, a model may indicate that purchase orders exceeding one million euros require two checks. Auditors can use conformance checking to detect deviations, locate and explain them, and measure their severity (A. Rozinat and W.M.P. van der Aalst, “Conformance Checking of Processes Based on Monitoring Real Behavior,” Information Systems, vol. 33, no. 1, 2008, pp. 64–95).

Toward Operational Support
Although process mining has traditionally focused on offline analysis and is seldom used for operational decision support, it can consider running-process instances and compare them with models based on historic data or business rules (W. M. P. van der Aalst, M. Pesic, and M. Song, “Beyond Process Mining: From the Past to Present and Future,” tech. report BPM-09-18 BPMcenter.org, 2009). For example, an auditor can “replay” a running case on the process model in real time and check whether the observed behavior fits. The moment the case deviates, the auditor can alert an appropriate actor.

Auditors can also use a process model based on historic data to make predictions about running cases—for example, to estimate the remaining processing time and a particular outcome's probability—or provide recommendations, such as an activity that will minimize the expected costs and completion time.

\section{Requirements}

% Anforderungen aufgrund von Gesprächen mit dem Anwender und aufgrund der technischen
% und sozialen Verantwortung des Entwicklers,Stand der Technik und ggf. der Forschung,

% Keine Anforderung darf im Folgenden vergessen werden (mindestens Erwähnung in "Weiter- entwicklungsmöglichkeiten").
% • Jedes Feature des konzipierten bzw. realisierten Systems muss auf eine Anforderung zurück- geführt werden.

The audit component implemented in chapter \ref{chap:ac} is to be utilised to offer an audit reporting \gls{ac:gui}. \gls{ac:m8} offer much more features, that can be utilised, thus the \gls{ac:gui} should be expandable to accommodate other use-cases and implementations.

Upstream \gls{ac:iam} authenticators like onelogin are to be supported.

Since Auditors have different backgrounds and technical knowledge the \gls{ac:ux} should be intuitive and well thought out. The following is to be considered:

\begin{itemize}
  \item The same event view musst be ensured to ease dealing with deferent events and allow for a sense of familiarity. For example if a table is used to showcase the event, the table structure is preserved for all types.
  \item lack of events is not an error and musst be clearly represented
\end{itemize}

To ease auditors workflows and external auditing needs audit reports should be exportable and sendable via e-mail in a spread-sheet compatible format like \gls{ac:csv}. Automating these reports should be possible as well.

% alerting and monitoring

% End with a GUI, that the auditor can configure to her/his own requirements

\subsection{Use-Cases}

\begin{table}[H]
  \begin{center}
    \caption{Audit browser derived use-cases}
    \label{tab:abuc}
    \def\arraystretch{1.5}
    \begin{tabularx}{\linewidth}{|l l X |}
      \hline
      \rowcolor{gray!20}
      \textbf{ID} & \textbf{Use-Case} & \textbf{Description}\\
      UC01 & \gls{ac:m8} \gls{ac:gui} & As a user, I want to use \gls{ac:m8} features through an easy to navigate \gls{ac:gui}\\
      UC02 & Onelogin \gls{ac:iam} authentication & As a user, I want to authenticate using my onelogin \gls{ac:iam} account\\
      UC04 & audit-log for date-range report & As an auditor, I want to see all actions taken within a specific date-range\\
      UC05 & audit-log about a user report & As an auditor, I want see all actions taken on a user\\
      UC06 & audit-log of user-actions report & As an auditor, I want see all actions taken by a user\\
      UC07 & audit-log users overview & As an auditor, I want a report of all users at a specific timestamp, tenants they belongs to, and their roles within the system or tenants/clusters\\
      UC08 & \gls{ac:csv} audit-log reports export & As an auditor, I want to export audit-log reports in a \gls{ac:csv} formatted file \gls{ac:gui}\\
      UC09 & Send audit-log reports via e-mail & As an auditor, I want to send audit-log reports via e-mail \gls{ac:gui}\\
      \hline
    \end{tabularx}
  \end{center}
\end{table}

% \subsection{Stakeholders}

\subsection{Architectural Constraints}

\subsubsection{Technical Constraints}

\begin{table}[H]
  \begin{center}
    \caption{Audit browser technical constraints}
    \label{tab:abtc}
    \def\arraystretch{1.5}
    \begin{tabularx}{\linewidth}{|l l X |}
      \hline
      \rowcolor{gray!20}
      \textbf{ID} & \textbf{Constraint} & \textbf{Description}\\
      TC01 & Expandable \gls{ac:gui} & \gls{ac:gui} implementation allow for other \gls{ac:m8} features support\\
      TC02 & Unified event view & Audit-log reports have unified event views\\
      \hline
    \end{tabularx}
  \end{center}
\end{table}

\subsubsection{Organisational Constraints}

\begin{table}[H]
  \begin{center}
    \caption{Audit component organisational constraints}
    \label{tab:acoc}
    \def\arraystretch{1.5}
    \begin{tabularx}{\linewidth}{|l l X |}
      \hline
      \rowcolor{gray!20}
      \textbf{ID} & \textbf{Constraint} & \textbf{Description}\\
      OC01 & Deadline & implementation musst be finalised before 28.02.2023\\
      \hline
    \end{tabularx}
  \end{center}
\end{table}

\section{System Design}

\subsection{Scope and Context}

\subsubsection{Business Context}

\includenamedsvg{\includesvg[width=\textwidth]}{diagrams/audit_browser/audit_browser_business_context.drawio}{Audit Browser Business Context Diagram}

\subsubsection{Technical Context}

\subsection{Solution Strategy}

\subsection{Building Block View}

\subsection{Runtime View}

\subsection{Deployment View}

\section{Design Decisions}

\subsection{DD01: gRPC proxy}

Communicating using the synchronous Remote
procedure invocation pattern \citep{richardson2018microservices}

comparison: https://youtu.be/CAGuhVIOT2c?t=1829

dealing with api changes

\subsection{DD02: Micro Frontends}

\section{Technical Decisions}

Lösungsalternativen sind kritisch zu bewerten und Auswahl zu begründen.

\subsection{Enabling Technologies}

the 8 fallacies of distributed computing

protobuf vs json: https://youtu.be/CAGuhVIOT2c?t=1206

Stand des Projekts und Weiterentwicklungsmöglichkeiten.
• Anhang: Entwicklerdokumentation,Benutzerdokumentation
Elektronischer Anhang: Sources, Projektdateien.