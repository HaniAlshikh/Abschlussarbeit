% @author Hani Alshikh
%
\chapter{Audit Browser}\label{chap:ab}

When it comes to staying complaint ascertaining the validity and reliability of information in organizations and their associated processes is a very important job of the organization security officers. It is also customary to have yearly audit sessions done by external entities. However, waiting for the audit report to hit should not be an option. Especially when it comes to highly regulated industries.

In chapter \ref{chap:adt} an example overview of the different internal auditing activities were showcased and discussed. The detective controls done by administrators like security officers rely on audit logs and the corresponding aggregation programs. Assuming a proper audit logging infrastructure there is still a dependency on the developer to properly log the relevant activities, which poses the responsibility to evaluate and reliably provide a consistent and complaint audit logs. Not only is this prone to human errors it is by design bond to to not provide the full picture and might lead to over logging. 

With time the audit log file just becomes larger and larger and going through the logs becomes very tedious and tend to be avoided. Even when it comes to external controls it is customary to work with pre-recorded parteitonend samples, which might give a good overview but never the full one. 

Auditing 2.0 discussed in section~\ref{sec:adt2} describe practices to mitigate the mentioned problems and provides the basis for an auditing frameworks in attempt of continues auditing approach. Using \gls{gl:es} as a reliable audit logging mechanism and the \gls{gl:ac} to implement the the underlying \gls{ac:api} all what is left is to provide a user friendly interface to the auditors while keeping the fact in mind, that most auditors have different technical backgrounds and some have none.

% This implementation attempts a basic \gls{ac:mvp} as a \gls{ac:poc} of the reading modules, that can be extended easily to handle actionable ones.

\section{Requirements}

% Anforderungen aufgrund von Gesprächen mit dem Anwender und aufgrund der technischen
% und sozialen Verantwortung des Entwicklers,Stand der Technik und ggf. der Forschung,

% Keine Anforderung darf im Folgenden vergessen werden (mindestens Erwähnung in "Weiter- entwicklungsmöglichkeiten").
% • Jedes Feature des konzipierten bzw. realisierten Systems muss auf eine Anforderung zurück- geführt werden.

The \gls{gl:ac} implemented in chapter \ref{chap:ac} is to be utilised to offer an audit log reporting \gls{ac:gui} to enable continues auditing and lay the ground work for \hyperref[sec:adt2]{Auditing 2.0}.

\Gls{ac:m8} offer much more features, that can be utilised. Since \gls{ac:m8} has no official \gls{ac:gui} the base implementation for the \gls{gl:ab} should be expandable to accommodate other use-cases and implementations.

Upstream \gls{ac:iam} providers like Onelogin are to be supported.

Since Auditors have different backgrounds and technical knowledge the \gls{ac:ux} should be intuitive and well thought out. The following is to be considered:

\begin{itemize}
  \item The same event view musst be ensured to ease dealing with different events and allow for a sense of familiarity. For example if a table is used to showcase the event, the table structure is preserved for all types.
  \item lack of events is not an error and musst be clearly represented
\end{itemize}

To ease auditors workflows and external auditing needs audit reports should be exportable and sendable via e-mail in a spread-sheet compatible format like \gls{ac:csv}. Automating these reports should be possible as well. \todo{see if this will be possible timewise}

\pagebreak

\subsection{Use-Cases}

\def\arraystretch{1.5}
\begin{xltabular}[H]{\linewidth}{| l l X |}
  \hline
  \rowcolor{gray!20}
    \textbf{ID} & \textbf{Use-Case} & \textbf{Description}\\
    UC01 & \Copy{abuc01}{Monoskope \gls{ac:gui}} & As a user, I want to use \gls{ac:m8} features through an easy to navigate \gls{ac:gui} \\

    UC02 & \Copy{abuc02}{\gls{ac:oidc} authentication} & As a user, I want to authenticate using my \gls{ac:oidc} provider account \\

    UC03 & \Copy{abuc03}{Audit-log for date-range report} & As an auditor, I want to see all actions taken within a specific date-range \\

    UC04 & \Copy{abuc04}{Audit-log about a user report} & As an auditor, I want see all actions taken on a user \\

    UC05 & \Copy{abuc05}{Audit-log of user-actions report} & As an auditor, I want see all actions taken by a user \\

    UC06 & \Copy{abuc06}{Audit-log users overview} & As an auditor, I want a report of all users at a specific timestamp, tenants they belongs to, and their roles within the system or tenants/clusters \\

    UC07 & \Copy{abuc07}{\gls{ac:csv} audit-log reports export} & As an auditor, I want to export audit-log reports in a \gls{ac:csv} formatted file \\

    UC08 & \Copy{abuc08}{Send audit-log reports via e-mail} & As an auditor, I want to send audit-log reports via Email \\
    \hline
  \caption{\Gls{gl:ab} derived use-cases\label{tab:abuc}}\\
\end{xltabular}

\subsection{Architectural Constraints}

\subsubsection{Technical Constraints}

\begin{table}[H]
  \begin{center}
    \label{tab:abtc}
    \def\arraystretch{1.5}
    \begin{tabularx}{\linewidth}{|l l X |}
      \hline
      \rowcolor{gray!20}
      \textbf{ID} & \textbf{Constraint} & \textbf{Description}\\
      TC01 & Expandable \gls{ac:gui} & \gls{ac:gui} implementation allow for other \gls{ac:m8} features support\\
      TC02 & Unified event view & Audit-log reports have unified event views\\
      TC03\label{tc:ab03} & Backend & \Gls{ac:m8} is written in \gls{gl:go}\\
      \hline
    \end{tabularx}
    \caption{\Gls{gl:ab} technical constraints}
  \end{center}
\end{table}

\subsubsection{Organisational Constraints}

\begin{table}[H]
  \begin{center}
    \label{tab:aboc}
    \def\arraystretch{1.5}
    \begin{tabularx}{\linewidth}{|l l X |}
      \hline
      \rowcolor{gray!20}
      \textbf{ID} & \textbf{Constraint} & \textbf{Description}\\
      OC01 & Deadline & Implementation musst be finalised before 24.02.2023\\
      OC02 & \gls{gl:12faktor} & Implementation musst adhere to the \gls{gl:12faktor} methodology\\
      OC03 & Userbase & Userbase is mainly desktop users\\
      \hline
    \end{tabularx}
  \end{center}
  \caption{\Gls{gl:ab} organisational constraints}
\end{table}

\section{System Design}

\subsection{Scope and Context}

\subsubsection{Business Context}

\includenamedsvg{\includesvg[width=\textwidth,pretex=\relsize{-10}]}{diagrams/audit_browser/audit_browser_business_context.drawio}{\Gls{gl:ab} business context diagram}

\subsubsection{Technical Context}

\includenamedsvg{\includesvg[width=\textwidth,pretex=\relsize{-10}]}{diagrams/audit_browser/audit_browser_technical_context.drawio}{\Gls{gl:ab} technical context diagram}

% https://youtu.be/XdG0EujccHQ?t=748

\subsection{Solution Strategy}


\begin{table}[H]
  \begin{center}
    \label{tab:abss}
    \def\arraystretch{1.5}
    \begin{tabularx}{\linewidth}{| X m{1cm} X X X |}
      \hline
      \rowcolor{gray!20}
      \textbf{Function} & \textbf{UCID} & \textbf{Semantics} & \textbf{Precondition} & \textbf{Postcondition}\\

      render(\-components) & UC01 & render components in the browser & user navigated to \gls{gl:monogui} based \gls{ac:uri} & requested components are rendered or user is redirected to authenticate himself \\

      signIn(\-m8APIUrl) & UC02 & Kick in \gls{ac:m8} \gls{ac:oidc} authentication flow & user clicked the sign in button & user is redirected to \gls{ac:idp} authentication \gls{ac:url} \\

      requestAuth(\-authCode) & UC02 & request authenticaton token from \gls{ac:m8} & user signed in using his \gls{ac:idp} account & user can access protected routes \\

      getAuditLog(\-from, to, kargs) & UC03 UC04 UC05 UC06 & get audit log depending on the usecase as described in \ref{tab:abuc} & user provided date range and other inputs based on the usecase & user can browse the requested log entries \\

      \hline
    \end{tabularx}
  \end{center}
  \caption{\Gls{gl:ab} solution strategy}
\end{table}

\subsection{Building Block View}

\subsubsection{Overall System White Box}

\includenamedsvg{\includesvg[width=12cm,pretex=\relscale{0.4}]}{diagrams/audit_browser/audit_browser_component_diagram_white_box_level_1.drawio}{\Gls{gl:ab} overall system component diagram}

\subparagraph{Level 1}

\header{Contained Building Blocks}

\header{MonoGUI}

\mytable{abmguibbbb}{
  \begin{tabularx}{\linewidth}{|l X |}
    \hline
    \rowcolor{gray!20}
    \textbf{Component} & \textbf{Description}\\
    gRPC & handles comunication with \gls{gl:grpcweb} Proxy\\
    \hline
  \end{tabularx}
}{\Gls{gl:ab} \gls{gl:monogui} contained building blocks black box}

\header{Scenes}

\mytable{abscbbbb}{
  \begin{tabularx}{\linewidth}{|l X |}
    \hline
    \rowcolor{gray!20}
    \textbf{Component} & \textbf{Description}\\
    Audit & handles rendering of the audit components and user inputs \\
    Auth & handles rendering of the authentication components and user inputs \\
    \hline
  \end{tabularx}
}{\Gls{gl:ab} scenes contained building blocks black box}

\pagebreak

\header{UseCases}

\mytable{abuccbbbb}{
  \begin{tabularx}{\linewidth}{|l X |}
    \hline
    \rowcolor{gray!20}
    \textbf{Component} & \textbf{Description}\\
    Audit & handles communication with \gls{ac:m8}'s \gls{gl:ac} and data preparation \\
    Auth & handles communication between \gls{ac:m8}'s Gateway, the user, the upstream \gls{ac:idp} \\
    \hline
  \end{tabularx}
}{\Gls{gl:ab} UseCases contained building blocks black box}

\header{API}

\mytable{abapicbbbb}{
  \begin{tabularx}{\linewidth}{|l X |}
    \hline
    \rowcolor{gray!20}
    \textbf{Component} & \textbf{Description}\\
    Audit & \gls{ac:m8}'s \gls{gl:ac} stubs to handel \gls{ac:grpc} requests \\
    Auth & \gls{ac:m8}'s Gateway stubs to handel \gls{ac:grpc} requests \\
    \hline
  \end{tabularx}
}{\Gls{gl:ab} API contained building blocks black box}

\pagebreak

\subparagraph{Level 2}

\header{\gls{gl:monogui} White Box}

\subheader{\gls{ac:grpc}}

\includenamedsvg{\includesvg}{diagrams/audit_browser/audit_browser_monogui_grpc_class_diagram.drawio}{\gls{gl:ab} \gls{gl:monogui} \gls{ac:grpc} class diagram}

\mytable{abmgcd}{
  \begin{tabularx}{\linewidth}{|l X |}
    \hline
    \rowcolor{gray!20}
    \textbf{Object} & \textbf{Description}\\
    GrpcConnectionFactory & creates preconfigured \gls{ac:grpc} connection based on the use case for example authenticated with timeout and retries connection \\
    \hline
  \end{tabularx}
}{\Gls{gl:ab} \gls{gl:monogui} \gls{ac:grpc} class diagram}

\pagebreak

\header{Scenes White Box}

\subheader{Audit}

\includenamedsvg{\includesvg[width=\textwidth,pretex=\relscale{0.6}]}{diagrams/audit_browser/audit_browser_scenes_audit_class_diagram.drawio}{\gls{gl:ab} scenes audit class diagram}

\mytable{absacd}{
  \begin{tabularx}{\linewidth}{|l X |}
    \hline
    \rowcolor{gray!20}
    \textbf{Object} & \textbf{Description}\\
    Sidebar & composition component for sidebar presentation and interaction \\
    Header & composition component for content header presentation and interaction \\
    Topbar & composition component for topbar presentation and interaction \\
    AuditDatePicker & composition component for date range presentation and interaction \\
    Audit & composed component for get by date range use case UC03~\ref{tab:abuc} presentation and interaction \\
    \hline
  \end{tabularx}
}{\Gls{gl:ab} scenes audit class diagram}

\subheader{Auth}

\includenamedsvg{\includesvg[width=\textwidth,pretex=\relscale{0.8}]}{diagrams/audit_browser/audit_browser_scenes_auth_class_diagram.drawio}{\gls{gl:ab} scenes auth class diagram}

\mytable{absacd}{
  \begin{tabularx}{\linewidth}{|l X |}
    \hline
    \rowcolor{gray!20}
    \textbf{Object} & \textbf{Description}\\
    ThemeButton & composition component for the sign in button presentation and interaction \\
    Auth & composed component for the authentication use case UC02~\ref{tab:abuc} presentation and interaction \\
    AuthContext & composition component to provide authentication context for composing components \\
    useAuth & composed component to manage authentication flow components \\
    AuthSecure & composition component to put composing components behind authentication wall \\
    useAuthenticatedClient & composition component to provide preconfigured and authenticated \gls{ac:grpc} client for composing components \\
    AuthPopup & composition component for upstream \gls{ac:idp} presentation and interaction \\
    \hline
  \end{tabularx}
}{\Gls{gl:ab} scenes auth class diagram}

\pagebreak

\header{UseCases White Box}

\subheader{Audit}

\includenamedsvg{\includesvg}{diagrams/audit_browser/audit_browser_usecases_audit_class_diagram.drawio}{\gls{gl:ab} usecases audit class diagram}

\mytable{abucacd}{
  \begin{tabularx}{\linewidth}{|l X |}
    \hline
    \rowcolor{gray!20}
    \textbf{Object} & \textbf{Description}\\
    UseCase & use case base for \gls{ac:api} aggregation and data preparation for presentation components \\
    GetAuditLogUseCase & request audit log events from the \gls{ac:api} and prepare them for presentation components \\
    \hline
  \end{tabularx}
}{\Gls{gl:ab} usecases audit class diagram}

\pagebreak

\subheader{Auth}

\includenamedsvg{\includesvg}{diagrams/audit_browser/audit_browser_usecases_auth_class_diagram.drawio}{\gls{gl:ab} usecases auth class diagram}

\mytable{abucathcd}{
  \begin{tabularx}{\linewidth}{|l X |}
    \hline
    \rowcolor{gray!20}
    \textbf{Object} & \textbf{Description}\\
    AuthUseCase & coordinate authentication flow initialisation and callbacks between the \gls{ac:api} and presentation components \\
    \hline
  \end{tabularx}
}{\Gls{gl:ab} usecases auth class diagram}

\subsection{Runtime View}

\subsubsection{\hyperref[tab:abuc]{UC01}: \Paste{abuc01}}

\includenamedsvg{\includesvg[width=\textwidth]}{diagrams/audit_browser/audit_browser_uc01_auth_sequence_diagram.drawio}{\gls{gl:ab} \hyperref[tab:abuc]{UC01} sequence diagram}

\subsubsection{\hyperref[tab:abuc]{UC02}: \Paste{abuc02}}

\includenamedsvg{\includesvg}{diagrams/audit_browser/audit_browser_uc02_auth_sequence_diagram.drawio}{\gls{gl:ab} \hyperref[tab:abuc]{UC02} sequence diagram}

\includenamedsvg{\includesvg[width=\textwidth,pretex=\relscale{0.6}]}{diagrams/audit_browser/audit_browser_uc02_auth_signin_sequence_diagram.drawio}{\gls{gl:ab} \hyperref[tab:abuc]{UC02} sign in sequence diagram}

\includenamedsvg{\includesvg[width=\textwidth,pretex=\relscale{0.6}]}{diagrams/audit_browser/audit_browser_uc02_auth_state_machine.drawio}{\gls{gl:ab} \hyperref[tab:abuc]{UC02} auth state machine}

\subsubsection{\hyperref[tab:abuc]{UC03}: \Paste{abuc03}}

\includenamedsvg{\includesvg}{diagrams/audit_browser/audit_browser_uc03_sequence_diagram.drawio}{\gls{gl:ab} \hyperref[tab:abuc]{UC03} sequence diagram}

\pagebreak

\section{Design Decisions}

\subsection{DD01: gRPC Client-Server Communication}

\gls{ac:http}/2 comes with multiple advantages to handle \gls{ac:http}/1.1 limitations like multiplexing, \gls{gl:hpack} compression and server push mechanism, which allows the server to push streams of messages without waiting for an explicit client request~\citep{RFC7540}

As of the time of writing Browsers do support \gls{ac:http}/2 but only for static files like images, javascript, css etc. XMLHttpRequest/Ajax calls are still carried over \gls{ac:http}/1.1. This is due to the fact, that browsers have no unified specification to handle \gls{ac:http}/2 trailers.

\includenamedsvg{\includesvg[width=\textwidth,pretex=\relscale{0.8}]}{diagrams/http1_vs_http2.drawio}{\gls{ac:http}/1.1 compared to \gls{ac:http}/2}

\gls{ac:grpc} makes heavy use of trailers to send status messages when streaming responses~\citep{grpcspec}. Because of the browser \gls{ac:http}/2 limitations \gls{ac:js}, the programming language of the web, can not directly talk to a \gls{ac:grpc} \glspl{ac:api} as normally done with \gls{ac:rest} \glspl{ac:api}. 

To mitigate this limitation a translation proxy has to be used to translate \gls{ac:grpc} to \gls{ac:rest} requests/responses~\citep{grpcBrowserState}.

\includenamedimage[0.5]{figures/grpc-web-proxy}{\gls{gl:grpcweb} proxy to allow browser \gls{ac:grpc} support~\citep{grpcBrowserState}}

\header{Why stick to gRPC?}

Deciding weather or not it is worth the effort to go through the hassle of proxying all requests to mitigate browser support and navigating through mainly an edge technology when it comes to web-development with no clear guides and mediocre documentation comes back to the magic phrase "it depends". the gained efficiency compared to \gls{ac:rest} and the maintainability efforts, especially for production usecases, requires thorough evaluation.

\begin{itemize}
  \item Messages passing back and forth are encoded which makes them hard to read and debug
  \item Production use is minimal at best since the effort is still not worth the gaines for already established services
  \item there is still no Client-side and Bi-directional streaming support
  \item official Protobuf compiler extensions are buggy and have no ECMAScript support (usage of third party extensions like protobuf-es is needed)
\end{itemize}

Despite the mentioned above \gls{ac:grpc} still shines when it comes to straightforward \gls{ac:api} definitions, efficient and compact mechanism to exchange big messages~\citep{richardson2018microservices}, streaming and \gls{ac:http}/2 advantages justify its usage in the right context. 

\header{Other Options}

The only other option is to actively provide backend \gls{ac:rest} support by Transcoding \gls{ac:http}/\gls{ac:json} to \gls{ac:grpc} and utilising a reverse proxy to to serve a \gls{gl:restfull} \gls{ac:api}~\citep{grpcgateway}

\includenamedsvg{\includesvg[width=\textwidth,pretex=\relscale{0.8}]}{figures/grpc-gateway_architecture_introduction_diagram}{\gls{ac:grpc}-Gateway architecture diagram~\citep{grpcgateway}}

While this comes with the \gls{gl:restfull} \gls{ac:api} advantages and full support of the web development community it dose not differ in it's essnase to \gls{gl:grpcweb} yet comes with the disadvantage of the server transcoding overhead.

\header{Conclusion}

From the perspective of using a technology that is on the bleeding edge of its peers just getting the basic workflows working with it vs. the well established patterns of \gls{ac:rest} is challenging.

However, since \gls{ac:m8} only speaks \gls{ac:grpc} and maintaining a \gls{ac:rest} \gls{ac:api} is not an options \hyperref[tab:actc]{TC03}, due to the fact that solutions do exists to to handle \gls{ac:rest}-\gls{ac:grpc} translation, until full \gls{ac:grpc} support by the browsers and the alternatives offers no real advantages in this case it was decided to go with \gls{ac:grpc} for client-server communication.

\section{Technical Decisions}

\subsection{TD01: \glsfirst{ac:js} and \gls{gl:react}}\label{abtd:01}

% JS/TS
% https://youtu.be/CAGuhVIOT2c?t=2449
% https://medium.com/zeals-tech-blog/grpc-go-server-web-client-853daf40cdef
% https://www.vinsguru.com/grpc-web-example/
% https://blog.cloudflare.com/moving-k8s-communication-to-grpc/
% react
% https://www.youtube.com/watch?v=NFZbTy_B4H0&t=423s
% https://programmingpercy.tech/blog/using-grpc-tls-go-react-no-reverse-proxy/
% https://www.youtube.com/watch?v=XdG0EujccHQ&t=3348s
% https://daily.dev/blog/build-a-chat-app-using-grpc-and-reactjs
% % https://medium.com/@mehran.hrajabi98/grpc-in-front-end-applications-a70a9cea4a43
% https://github.com/mehran-hrajabi/react-grpc-boilerplate
% https://medium.com/alva-labs/building-microapps-with-grpc-web-64b7cdf50313
% https://www.polarsignals.com/blog/posts/2022/06/23/how-we-use-gRPC-to-build-our-frontend/
% https://www.youtube.com/watch?v=NFZbTy_B4H0&t=1944s
% Vue.js
% https://medium.com/@itachisasuke/using-grpc-web-vue-js-with-a-golang-backend-76dd9d6fd150
% https://gustavohenrique.net/en/2020/02/grpc-web-with-golang-and-vuejs-an-alternative-to-rest-and-graphql/
% https://torq.io/blog/grpc-web-using-grpc-in-your-front-end-application/
% https://github.com/kostyay/grpc-web-example
% Angular
% https://www.youtube.com/watch?v=fvFOBkR6GT8&t=1139s
% https://github.com/search?q=user%3Avishipayyallore+grpc-web&type=code
% https://itnext.io/a-complete-guide-to-grpc-web-with-angular-and-net-c4ae2500bd24
% WASM
% https://www.youtube.com/watch?v=swZPL5PjkVU&t=187s
%  youtube.com/watch?v=wUCCrrlzkPU&t=34s
% https://www.youtube.com/watch?v=FBRbdpdK3lI&t=155s

It is no secret that \gls{ac:js} is the language of the web and is almost always ranked first by multiple indexes compared to other web development programing languages.

\includenamedimage[0.7]{figures/Top-Programming-Languages-comparison-table}{Top programming languages - rankings in comparison \citep{ProgrammingLangRank}}

In a survey conducted on 39,472 person a clear trend towards \gls{ac:ts} can be seen as it improves the developer experience and provides various other features mainly based on the added typing support \citep{TheStateOfJs}

However writing vanilla \gls{ac:js} or \gls{ac:ts} tend to be very cumbersome and follow a scripting approach (hence the name). Depending on the use-case this might be much preferable, sense it offer total control and a declarative approach when it comes to handling \gls{ac:xml} and \gls{ac:html} elements. 

On the other hand building and maintaining web applications requires a programming approach and patterns to achieve production level results, which lead to the development of many frontend development frameworks lke \gls{gl:react}, \gls{gl:angular}, \gls{gl:vuejs} and many others.

Among the 3 stables mentioned above \gls{gl:react} has the higher usage across all years since 2016 \citep{TheStateOfJs} since it comes with many basic yet powerful features, that can be expanded depending on the use-case. \gls{gl:angular} on the other hand tend to be tailored towards enterprise usage and comes with greater complexity and steeper learning curve, while \gls{gl:vuejs} attempt to be the happy medium in between.

\header{Conclusion}

While \gls{ac:ts} seems to be the smart choose to make, especially for production environments the added complexity requires more development time. Since \gls{gl:monogui} is meant to be a \gls{ac:poc} and is for now mainly developed as a base for the \gls{gl:ab} and due to the deadline set by \hyperref[tab:aboc]{OC01} it was decided to go with \gls{ac:js} in combination with \gls{gl:react} since they offer the perfect platform for the planned solution.

\subsection{TD02: Headless Use-Cases Implementation}

While \hyperref[abtd:01]{TD01} specifies usage of \gls{ac:js} and \gls{gl:react} \gls{gl:monogui} expansion and further development should be kept in mind as specified by \hyperref[tab:abtc]{TC01}. 

To stay flexible and framework independent it was decided to split the business logic from the presentation logic or as is commonly known follow a headless implementation approach. The latter specifies writing business logic in vanilla \gls{ac:js} and the presentation logic in the used framework notation. This allows for easier migration to \gls{gl:ts} while also staying framework independent. 

% \subsection{TD01: gRPC Web}

% https://grpc.io/blog/grpc-web-ga/

% gRPC REST gateway implementation.

% options:

% - gRPC Gateway:
% --- server need to know about it
% --- -> more maintenance cost
% --- -> basically REST API
% % https://jbrandhorst.com/post/gopherjs-grpcweb/
% % https://github.com/TanStack/query/discussions/978#discussioncomment-741259
% Clients don’t know what types are used - the interface is HTTP JSON. This can be somewhat mitigated with the use of swagger generated interfaces, but it’s still not perfect.
% The interface being JSON means marshalling and unmarshalling can become a significant part of the latency between the client and the server.
% The gRPC-gateway requires specific changes to the proto definitions - it’s not as straightforward as just defining your RPC methods.

% - gRPC Web: 
% ---- only envoy support it
% ---- still experimental
% ---- only support unary and server streaming

% -- Connect-web

% since m8 is not a browser first system maintaining the connect protocol is not justified. 

% Connect for Web is currently pre-v1, but we have been using it in production for a while. We plan to release a stable v1 in early 2023. However, you can get started with connect-web now.

% connect-web uses protoc-gen-es, which ommit service generation in faivour of serviceType and reflection... % https://github.com/bufbuild/protobuf-es/issues/252#issuecomment-1282549160
% This generally requires deeper knowledge and make a bit tedious and cumbersome for contributors. 

% -- protobuf-ts

% is the root project that led the seeds for connect.... bla bla bla
% it support service initializers

% -- nice-grpc

% nice grpc is an alternativ typescript native implementation, which looks promising but....

% loadbalancing:

% - nginx

% - envoy
% \gls{ac:m8} comes with \gls{gl:ei} which is built on top of \gls{gl:envoy}
% .Net seems have the most adaption and support for web-gRPC but due to the backend being written in \gls{gl:go} (TC03~ref{tc:ab03}) ...

% \subsection{TD02: React}
% % https://github.com/grpc/grpc-web/blob/master/doc/roadmap.md#web-framework-integration

% Angular for UI heavy -> vue in the middle -> react for casual use

% Anguler tend to be overkill for small projects and monoskope is not GUI intesive produckt
