% @author Hani Alshikh
%
\chapter{Audit}\label{chap:adt}

An audit is often defined as an independent examination, inspection, or review. While the term applies to evaluations of many different subjects, the most frequent usage is with respect to examining an organization’s financial statements or accounts. Words like assessment, evaluation, and review are often used synonymously with the term audit and while it is certainly true that an audit is a type of evaluation, some specific characteristics of auditing distinguish it from concepts implied by the use of more general terms. 

An audit always has a baseline or standard of reference against which the subject of the audit is compared. An audit is not intended to check on the use of best practices or to see if opportunities exist to improve or optimize processes or operational characteristics. Instead, there is a set of standards providing a basis for comparison established prior to initiating the audit. \citep{GANTZ20141}

Audit determinations tend to be more binary than results of other
types of assessments or evaluations, in the sense that a given item either meets or fails to meet applicable requirements. Auditors often articulate audit findings in terms of controls conformity or nonconformity to criteria.

The \gls{ac:iso} guidelines on auditing use the term audit to mean: \begin{quote} A systematic, independent and documented process for obtaining objective evidence and evaluating it objectively to determine the extent to which the audit criteria are fulfilled \citep{ISOISO1953}\end{quote}

In contrast to conventional dictionary definitions and sources focused on
the accounting connotation of audit, definitions used by broad-scope audit standards bodies and in \gls{ac:it} auditing contexts neither constrain nor presume the subject to which an audit applies.

Such general interpretations are well suited to \gls{ac:it} auditing, which comprises a wide range of standards, requirements, and other auditing criteria to audit \gls{ac:it} subjects.

\section{\gls{ac:it} Audit}\label{sec:it-audit}

\gls{ac:it} audit is the process of collecting and evaluating evidence of an organization's \gls{ac:it} systems, practices, and operations to determine whether they are adequate, efficient, and effective in meeting the organization's objectives. \citep{GANTZ20141}

An \gls{ac:it} audit typically includes a review of an organization's policies, procedures, and controls related to its \gls{ac:it} systems, as well as an assessment of the security, reliability, and performance of its \gls{ac:it} infrastructure. The goal of an \gls{ac:it} audit is to identify any weaknesses or deficiencies in an organization's \gls{ac:it} systems and recommend improvements that can help the organization achieve its goals.

\includenamedimage[0.6]{figures/it_auditing_and_other_types.png}{\gls{ac:it} auditing commonality with other types of audit \citep{GANTZ20141}}

\gls{ac:it} auditing has much in common with other types of audit and overlaps in many respects with financial, operational, and quality audit practices. It is important to use “\gls{ac:it}” to qualify \gls{ac:it} audit and distinguish it from the more common financial connotation of the word audit used alone. 

\section{\gls{ac:it} Auditing in Context}

From the perspective of planning and performing IT audits, controls represent the substance of auditing activities, as the controls are the items that are examined, tested, analyzed, or otherwise evaluated \citep{GANTZ20141}.

\includenamedimage[0.9]{figures/it_audit_activities}{IT audit activities and scopes \citep{GANTZ20141}}

IT audits are performed both by internal auditors working for the organization and external auditors hired by it. The processes and procedures followed in internal and external auditing are often quite similar

The following is a one of the wide spread controls categorization schemes used in internal control frameworks. Controls are normally classified by purpose, functional type, or both.

\mytable{eicctp}{
  \begin{tabularx}{\linewidth}{|l X X X |}
    \hline
    \rowcolor{gray!20}
    \textbf{} & \textbf{Preventive} & \textbf{Detective} & \textbf{Corrective}\\
    Administrative & Acceptable use policy; Security awareness training & Audit log review procedures; IT audit program & Disaster recovery plan; Plan of action and milestones\\
    Technical & Application firewall; Logical access control & Network monitoring; Vulnerability scanning & Incident response center; Data and system backup\\
    Physical & Locked doors and server cabinets; Biometric access control& Video surveillance; Burglar alarm & Alternate processing facility; Sprinkler system\\
    \hline
  \end{tabularx}
}{Internal controls categorized by type and purpose \citep{GANTZ20141}}

Administrative controls specify what an organization intends to do to safeguard the integrity of its operations, information, and other assets.

IT audits and the approaches used to conduct them may consider internal controls from multiple perspectives by focusing on different IT elements. For now the focus will lay on audit log reviews and audit logging.

\pagebreak

\section{Audit Log}\label{sec:adtlog}

The purpose of audit logging is to record each state change. An audit log is typically used to help customer support, ensure compliance, and detect suspicious behaviors. Each audit log entry records at minimum the identity of the entity, the action performed, and the business object(s)~\citep{richardson2018microservices}

Depending on the requirements maintaining and ensuring a comprehensive and immutable Audit Log is mandatory and might dictate the way a system is architected and developed. An example will be showcased in chapter \ref{chap:ac}.

The Federal Financial Supervisory Authority (short BaFin in german) requires appropriate precautions to be taken within the framework of application development, so that the confidentiality, integrity, availability and authenticity of the data to be processed are transparently ensured even after each deployment of an application \citep{BaFinZAIT}. 

One of the appropriate precautions suggested by \gls{gl:bafin} is audit logs. Audit logging is not only a suggestion but also an indirect requirement:

\begin{quote}
In accordance with the target protection requirements the institution must set up processes for logging and monitoring, which make it possible to verify, that authorizations are only used as intended \citep{BaFinZAIT}.
\end{quote}

An Audit Log can take many forms. The most common form is a file. A database table is also an option. However most problems comes mainly from the kind of logs written and the way they are processed. More on this is discussed in chapter \ref{chap:sadt}

Audit Log is easy to write but harder to read, especially as it grows large. Occasional ad-hoc reads can be done by eye and simple text processing tools. More complicated or repetitive tasks can be automated \citep{AuditLog}. Audit log entries lack the context and describe an action, that may or may not be related to other entries in a specific period of time. Keeping track of changes without proper automation becomes harder and harder to the point it becomes imposable.

For some organizations logging system changes alone is not enough. Suspicious activates might originate from read attempts. Logging such activities increases the complexity of the Audit Log, violates the Audit Log definition and poses the question of how to make sense of such data?

\section{Auditing 2.0}\label{sec:adt2}

Auditing 2.0, also known as continuous auditing, is a modern approach to auditing that uses technology and data analytics to continuously monitor and assess an organization's processes. Unlike traditional auditing, which is typically conducted on a periodic basis, continuous auditing is a continuous process that uses, but not limited to, real-time data to identify and address potential risks and issues as they arise.~\citep{5427384}

Auditing 2.0 makes use of technologies such as \gls{ac:ai} to automate and streamline the audit process. For example, AI-powered systems can be used to analyze and interpret data in real-time, while patterns like \gls{gl:es} provide a transparent and immutable record of events. 

\subsection{Business Provenance}\label{sec:bsprov}

The systematic, reliable, and trustworthy recording of events, known as business provenance, is essential to auditing in general and \gls{gl:adt2} in particular. This term acknowledges the importance of traceability by ensuring that history cannot be rewritten or obscured \citep{5427384}.

Traditionally, an audit can only provide reasonable assurance that business processes are executed within the given set of boundaries. Auditors assess the operating effectiveness of process controls, and when these controls are not in place or functioning as expected, they typically check samples of factual data. However, with detailed information about processes increasingly available in high-quality event logs, auditors no longer have to rely on a small set of samples offline. Instead, using process mining techniques, they can evaluate all events in a business process, and do so while it is still running.

\subsection{Process Mining}\label{sec:psmin}

The goal of process mining is to discover, monitor, and improve real (not assumed) processes by extracting knowledge from event logs. 

Process mining starts with the event log: a sequentially recorded collection of events, each of which refers to an activity (well-defined step) and is related to a particular case (process instance). Some mining techniques use other information such as the person or resource initiating the activity, the event's timestamp, or data elements recorded with the event \citep{5427384}.

Auditors can use process mining techniques to evaluate all events in a business process, and do so while it is still running. With the help of \gls{ac:ai} potential compliance violation and suspicious behaviour can be be detected while in the making and prevented before even happening. Reliable information is needed to determine whether these processes are executed within certain boundaries set by managers, governments, and other stakeholders. 