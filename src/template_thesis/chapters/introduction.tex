% @author Hani Alshikh
%
\chapter{Introduction}

% Fangen Sie nicht zu früh mit der Ausformulierung des Textes an!
% Benutzen Sie die Gliederungsfunktion handelsüblicher Textsysteme, um Ihre Gliederungspunkte ent- sprechend ihrer Bedeutung verschieben zu können. Füllen Sie die Textkomponenten zu Ihren Gliede- rungspunkten nach und nach erst mit Stichworten und Bildverweisen, später mit formulierten Sät- zen. Je besser dies gelingt, desto einfacher wird später die Ausformulierung des Textes. Damit haben Sie ein flexibles Ablagesystem für Ideen und Formulierungen.

% Aufgabenstellung (Problembeschreibung inkl. Motivation, Zielsetzung, Abgrenzung),

% Die Einleitung ist das erste fachliche Kapitel einer Arbeit. Hier machen Sie den Leser mit der Problem- stellung und dem Umfeld bekannt, stellen das benötigte Basis- und Umgebungswissen zur Verfügung, nehmen Bezug auf andere Arbeiten und führen so in die Problemstellung ein. Hier findet sich u.a. gewiss wieder, was Sie zu Beginn Ihrer Tätigkeit im Pflichtenheft für die Bachelorarbeit geschrieben haben, nach entsprechender Überarbeitung und Anpassung, versteht sich. Bedenken Sie, dass der normale Leser zwar allgemeines Informatik-Fachwissen besitzt, aber nicht das konkrete Umfeld ihrer Firma/Arbeit kennt bzw. die dort vorhandene Begriffswelt. Hiermit müssen Sie ihn vertraut machen sowie Ihre spezielle Zielsetzung verdeutlichen und abgrenzen.
% Die Einleitung sollte in mehrere Abschnitte unterteilt sein, z. B. 1.1 Problemstellung und Motivation
% 1.2 Zielsetzung
% 1.3 Abgrenzung des Themas
% 1.4 Überblick über den Aufbau der Arbeit (Kapitelübersicht)

% Am Ende der Einleitung sollte der Leser mit dem Umfeld, der Problemstellung und der Zielsetzung in Umrissen vertraut sein.

Audits are systematic and objective examinations of one or more aspects of an organization, that compares what the organization does to a defined set of criteria or requirements. \gls{ac:it} auditing examines processes, \gls{ac:it} assets, and controls at multiple levels within an organization to determine the extent to which the organization adheres to applicable standards or requirements. 

\gls{gl:es} is a software architecture pattern that insures a complete log of changes made to a system as a series of events. Instead of storing the current state in a traditional database, \gls{gl:es} stores the history of changes made over time. This allows developers to rebuild the current state at any point in time and see exactly how it changed by replaying the stored events, which is very useful for debugging and performing rollbacks or reversals of changes.

Having a comprehensive and immutable audit trail makes \gls{gl:es} particularly well-suited for systems with complex business processes, that need to track and audit changes to sensitive data and ensure they are in compliance with regulations and standards without relying on traditional logging mechanisms.

In addition to providing a detailed audit trail, \gls{gl:es} also offers a number of other benefits. Since the events are stored in a chronological order, it is possible to implement time-based queries and manipulations, which lays the base for Auditing 2.0.

Beside evaluating \gls{gl:es} in regards to auditing and audit controls this work provides a \gls{ac:poc1} implementation of an \gls{gl:ab} for a \gls{ac:k8s} multi-cloud multi-cluster authentication and authorization system, that implements the base structure for further \gls{gl:adt2} implementations as well as the \gls{gl:ac} to serialize the event log into an Audit Log and offer the \gls{ac:api} for various clients utilisation.