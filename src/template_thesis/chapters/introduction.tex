% @author Hani Alshikh
%
\chapter{Introduction}

Fangen Sie nicht zu früh mit der Ausformulierung des Textes an!
Benutzen Sie die Gliederungsfunktion handelsüblicher Textsysteme, um Ihre Gliederungspunkte ent- sprechend ihrer Bedeutung verschieben zu können. Füllen Sie die Textkomponenten zu Ihren Gliede- rungspunkten nach und nach erst mit Stichworten und Bildverweisen, später mit formulierten Sät- zen. Je besser dies gelingt, desto einfacher wird später die Ausformulierung des Textes. Damit haben Sie ein flexibles Ablagesystem für Ideen und Formulierungen.

Aufgabenstellung (Problembeschreibung inkl. Motivation, Zielsetzung, Abgrenzung),

Die Einleitung ist das erste fachliche Kapitel einer Arbeit. Hier machen Sie den Leser mit der Problem- stellung und dem Umfeld bekannt, stellen das benötigte Basis- und Umgebungswissen zur Verfügung, nehmen Bezug auf andere Arbeiten und führen so in die Problemstellung ein. Hier findet sich u.a. gewiss wieder, was Sie zu Beginn Ihrer Tätigkeit im Pflichtenheft für die Bachelorarbeit geschrieben haben, nach entsprechender Überarbeitung und Anpassung, versteht sich. Bedenken Sie, dass der normale Leser zwar allgemeines Informatik-Fachwissen besitzt, aber nicht das konkrete Umfeld ihrer Firma/Arbeit kennt bzw. die dort vorhandene Begriffswelt. Hiermit müssen Sie ihn vertraut machen sowie Ihre spezielle Zielsetzung verdeutlichen und abgrenzen.
Die Einleitung sollte in mehrere Abschnitte unterteilt sein, z. B. 1.1 Problemstellung und Motivation
1.2 Zielsetzung
1.3 Abgrenzung des Themas
1.4 Überblick über den Aufbau der Arbeit (Kapitelübersicht)

Am Ende der Einleitung sollte der Leser mit dem Umfeld, der Problemstellung und der Zielsetzung in Umrissen vertraut sein.

Event sourcing is a software architecture pattern that involves storing a log of changes made to an application's data as a series of events. Instead of storing the current state of the data in a traditional database, event sourcing stores the history of changes made to the data over time. This allows developers to rebuild the current state of the data at any point in time by replaying the events in the log.

One of the primary benefits of event sourcing is that it provides a complete audit trail of all changes made to the data. Because the log of events is comprehensive and immutable, it is possible to trace the history of any given piece of data and see exactly how it has changed over time. This can be extremely useful for auditing purposes, as it allows organizations to track and monitor changes to sensitive data and ensure that they are in compliance with regulations and standards.

In addition to providing a detailed audit trail, event sourcing also offers a number of other benefits. Because the events in the log are stored in a chronological order, it is possible to use event sourcing to implement time-based queries and to reconstruct the state of the system at any point in the past. This can be useful for debugging and for performing rollbacks or reversals of changes.

Event sourcing is particularly well-suited for applications that need to handle large volumes of data and that require a high degree of data integrity. It is also a good fit for applications that need to support complex business processes and that require the ability to track and audit changes to the data over time.

Overall, event sourcing is an ideal pattern for auditing because it provides a complete and immutable record of all changes made to the data, making it possible to trace the history of any piece of data and ensure compliance with regulations and standards.
