% @author Hani Alshikh
%
\chapter{Introduction}

Audits are systematic and objective examinations of one or more aspects of an organization, that compares what the organization does to a defined set of criteria or requirements. \gls{ac:it} auditing examines processes, \gls{ac:it} assets, and controls at multiple levels within an organization to determine the extent to which the organization adheres to applicable standards or requirements. 

\gls{gl:es} is a software architecture pattern that insures a complete log of changes made to a system as a series of events. Instead of storing the current state in a traditional database, \gls{gl:es} stores the history of changes made over time. This allows developers to rebuild the current state at any point in time and see exactly how it changed by replaying the stored events, which is very useful for debugging and performing rollbacks or reversals of changes.

Having a comprehensive and immutable audit trail makes \gls{gl:es} particularly well-suited for systems with complex business processes, that need to track and audit changes to sensitive data and ensure they are in compliance with regulations and standards without relying on traditional logging mechanisms.

In addition to providing a detailed audit trail, \gls{gl:es} also offers a number of other benefits. Since the events are stored in a chronological order, it is possible to implement time-based queries and manipulations, which lays the base for Auditing 2.0.

Beside evaluating \gls{gl:es} in regards to auditing and audit logging this work make use of a real world use-case of audit logging needs by implementing the \gls{gl:ac} to serialize the event log into an audit log and offer an audit \gls{ac:api} for a \gls{ac:k8s} multi-cloud multi-cluster authentication and authorization system used by a \gls{gl:fintech} organization as well as an \gls{gl:ab} to ease the auditors interactions with the system and help answers various compliance questions.