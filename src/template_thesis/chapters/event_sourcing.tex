% @author Hani Alshikh
%
\chapter{Event Sourcing}\label{chap:es}

\gls{gl:es} is a software architecture pattern that was originally established by \citep{fowleres} and is gaining popularity as an alternative to traditional database systems. \gls{gl:es} stores data in an append-only log. It is part of a wider ecosystem of design patterns that work together in various ways to allow developers to create the most effective architecture for their needs. 

Traditionally applications use the current state to answer various queries, however in most cases such quires fails, when the path leading to the current state is required.

\gls{gl:es} ensures that all changes to application state are stored as a sequence of events. Writing an event to the log is one single, therefore atomic, operation. These events can be aggregated in multiple ways not only can they be quired, reconstructing past states is also possible. \gls{gl:es} can be thought of as the version-control way of working with system's state. Each change is committed and can be traced back and/or revisited.

\gls{gl:es} is a superior pattern for auditing compared to other architectures. Just like accounting transactions, events are never deleted nor modified. \gls{gl:es} provides a complete and immutable record of all actions taken within a system, allowing for thorough and accurate auditing. Unlike traditional architectures that rely on snapshots of data at a specific point in time, \gls{gl:es} captures every individual event and action, providing a more comprehensive and transparent record. This allows for more effective detection and investigation of potential issues or irregularities, ensuring the integrity and reliability of the auditing process. Additionally, the use of \gls{gl:es} allows for greater flexibility and scalability in auditing, as it allows for easy replay and reconstruction of past states. Alternative histories can also be explored by injecting hypothetical events when reconstructing the state.

\section{Terminology}\label{sec:term}

\subsection{The Event in Event-Sourcing}

While \gls{gl:es} and Event-Driven might sound similar they differ in multiple aspects. The events in \gls{gl:es}, as opposed to general event-driven architectures are stored as an append-only log of all state changes. The following are the two main key characteristics as derived by \citep{esvsed} separating \gls{gl:es} from other event-driven approaches.

\begin{enumerate}
  \item Events in \gls{gl:es} systems are stored as the state of the system. Other approaches use events to communicate and sometimes to passively send commands to the recipients. While the communication aspect in \gls{gl:es} is present, it fail second to the usage as the state.
  \item The second characteristic is, that events in \gls{gl:es} are used as the source of truth. Unlike other patterns that uses events to carry state changes. \Gls{gl:es} uses the events to derive the state change. 
\end{enumerate}

An event is a class with a name formed using a past-participle verb. It has properties that meaningfully convey the event. Each property is either a primitive value or a value object~\citep{richardson2018microservices}. Events typically also have metadata, such as the event ID and a timestamp. The metadata can be part of the event object or, alternatively, in a separate envelope object. It might also have the identity of the entity who made the change. Such details will come in handy when utilising process mining as described in section~\ref{sec:psmin}. Examples of such events are showcased later in chapter \ref{chap:ac}.

\subsection{Event Store}\label{subsec:est}

The event store is the database storing the events. The event store is the principal source of truth, and the system state is purely derived from it~\citep{esvsed}. Thus the database musst satisfy the following criteria:

\begin{itemize}
  \item Events are immutable.
  \item New events are appended to the previous event.
  \item Events are stored in chronological order.
\end{itemize}

\subsection{Streams}

The set of events comprising a particular domain object are called a stream. Event streams are the the source of truth of all domain objects in a system and contain the full history of changes. Retrieving the state of a domain object consist of reading all events in a stream and applying them one by one in the order of appearance.

A stream have a unique identifier present in all corresponding events. Detecting concurrency issues and insuring ordinality require uniq numerical value, that can be used as a form of versioning.

\subsection{Projections}

Projections provide a view of the underlying stream as a form of a transient state. They represent the logic of translating the source events into a representation of the object state.

In many applications it is more common to request recent application states, if so, a faster alternative is to store the current application projections and upsert on new changes.

\subsection{Snapshots}\label{subsec:snap}

snapshots create a working copy of a state that can be updated without replaying all events from scratch every time.

A common mistake is the assumption, that accessing the \gls{gl:est} is required to rebuild the state on each change. Accessing the \gls{gl:est} should generally be reserved to determining useful information. Only elements that really need the information in the event log should have to access it~\citep{esvsed}.

Snapshots can also be used to mark a state just like a tag in version-control systems. Depending on the use-case, this becomes helpful if the event log became too large in size.

\section{The Core Pattern}\label{sec:escp}

The fundamental idea of \gls{gl:es} is that of ensuring every change to the state of a system is captured in an event object, and that these event objects are themselves stored in the sequence they were applied for the same lifetime as the system state itself.

This leads to a number of facilities that can be built on top of the event log~\citep{fowleres}:

\begin{itemize}
  \item Complete Rebuild: the system state can be discarded completely and rebuilt by re-running the events from the event log.
  \item Temporal Query: the application state at any point in time can be determined. Notionally this is done by starting with a blank state and rerunning the events up to a particular time or event.
  \item Event Replay: the consequences of a corrupting event can be computed by reversing it and later events and then replaying the new event and later events. The same technique can handle events received in the wrong sequence - a common problem with systems that communicate with asynchronous messaging.
\end{itemize}

A common example of an application that uses \gls{gl:es} is version control. Such a system uses temporal queries quite often~\citep{fowleres}. Recently, the \gls{gl:es} pattern has become a popular answer to the challenges of complex, mission-critical, scalable systems~\citep{OVEREEM2021110970}. Examples of organizations that apply \gls{gl:es} are Netflix~\citep{avery2017scaling}, and Walmart’s Jet.com~\citep{jet2017scaling}, with the goal of creating scalable and reliable critical systems.

The main introduction to the inner workings of \gls{gl:es} by \citep{fowleres} gives a clear impression on the general implementation of the different building blocks of the pattern. More details and advanced concepts are covered by~\citep{richardson2018microservices}.

\pagebreak

\section{Event-Sourced Architecture}\label{sec:esa}

Using \gls{gl:es} as a design pattern within a wider architecture allows for the inclusion of other design patterns in the system that are the most suitable for the needs of the domain. For example, \gls{ac:ddd} in combination with \gls{gl:es} and \gls{ac:cqrs} lay the basis for a scalable architecture, that can be used in a variety of systems or in conjunction with many other patterns depending on the specific needs.

\header{\glsfirst{ac:ddd}}

Using \gls{ac:ddd} with \gls{gl:es} is not mandatory. However, in \gls{gl:es} events are modeled as first-class objects and closely resemble real world business processes. The better business processes are understood, the more precise the business information will be in the events and thus the audit log. Concepts like speaking the same language as the business and using events as a design tool when modelling a system are advocated by \gls{ac:ddd}~\citep{evans2004domain} as well, which makes \gls{gl:es} a natural fit for \gls{ac:ddd}.

\header{\glsfirst{ac:cqrs}}

\gls{gl:es} is discussed in the context of \gls{ac:cqrs}, a pattern strongly related to \gls{gl:es}.

\begin{quote}
\gls{ac:cqrs} and \gls{gl:es} have a symbiotic relationship. \gls{ac:cqrs} allows \gls{gl:es} to be used as the
data storage mechanism for the domain. One of the largest issues when using \gls{gl:es} is that you
cannot ask the system a query such as “Give me all users whose first names are ‘Greg’”. This is due to
not having a representation of current state. With \gls{ac:cqrs} the only query that exists within the domain is
GetById which is supported with \gls{gl:es}~\citep{young2010cqrs}.
\end{quote}

The loosely coupled nature of \gls{ac:cqrs} combined with the benefits of the \gls{gl:es} approach makes it a fitting architectural pattern for cloud systems. \gls{gl:es} itself is not tied exclusively to \gls{ac:cqrs}, the coupling based on events is similar to that in more general event-driven architectures.

The justification for \gls{ac:cqrs} is that in complex domains, a single model to handle both reads and writes gets too complicated, and can be simplified by separating the models. This is particularly appealing when difference in access patterns is observed, such as lots of reads and very few writes. However the gain for using \gls{ac:cqrs} has to be balanced against the additional complexity of having separate models~\citep{esvsed}.

While using \gls{ac:cqrs} is also not mandatory the strong relation and added benefits do justify the cost. Aggregating the \gls{gl:est} to satisfy the \gls{gl:ac} use-cases is an example, where \gls{ac:cqrs} shines. More on this in chapter \ref{chap:ac}

\section{Challenges}

\gls{gl:es} does have its problems. Replaying events becomes problematic when results depend on interactions with outside systems. Dealing with changes in the schema of events over time is not an easy task and event processing adds complexity to the system (mostly when improperly done)~\citep{esvsed}.

\subsection{Event Storage}

\gls{gl:es} enables the reconstruction of arbitrary past states. However, an entirely unbounded log size can conflict with other system requirements. As discussed in section \ref{subsec:snap}, snapshots offer a mitigation when the history is not relevant anymore for further processing as for example described in section \ref{sec:psmin}.

Other approaches like log pruning are discussed by \citep{10.1145/3210284.3219767} including an assessment of the impact of such mechanisms on state reconstructibility.

\subsection{Event Schema Evolution}

With \gls{gl:es}, the schema of events (and snapshots) will evolve over time. Because events are stored forever, business objects potentially need to fold events corresponding to multiple schema versions. There is a real risk that the objects may become bloated with code to deal with all the different versions~\citep{richardson2018microservices}. 

Upgrading events to the latest version when they are loaded from the \gls{gl:est} insures, that the system only ever deals with the current event schema. A component commonly called an \sh{upcaster} as described by~\citep{richardson2018microservices} updates individual events from an old version to a newer one.

\subsection{Deleting Data is Tricky}

One of the key characteristics of \gls{gl:es} is the immutable event log. The traditional way to delete data is to do a soft delete~\citep{richardson2018microservices}. The system deletes an object by setting a deleted flag. The object will typically emit a deleted event, which notifies any interested consumers. Any code that accesses that object can check the flag and act accordingly.

However complying with \gls{ac:gdpr} grants individuals the right to erasure~\citep{Art17GDP3}. An application must have the ability to forget a user’s personal information. One way of doing so is to ensure, that user data are encapsulated in an independent data object, that can either be encrypted by per user encryption key, which is discarded on request or iteratively overwritten, which is against the immutability aspect of the event.

Some form of anonymization and removal of information are two techniques mentioned by the engineers of the study conducted by \citep{OVEREEM2021110970}. The system separates the events and the personal information in two different stores. When events are read, they are supplemented with the personal information. If that information is no longer present (because of removal requests), default values are supplied.

\subsection{Querying the \gls{gl:est} is Challenging.}

As discussed in section \ref{sec:esa} implementing the core \gls{gl:es} pattern alone comes with the challenge of query complications. As put by~\citep{richardson2018microservices}:

\begin{quote}
  Imagine you need to find customers who have exhausted their credit limit. Because there isn’t a column containing the credit, you can’t write \sh{SELECT * FROM CUSTOMER WHERE CREDIT_LIMIT = 0}. Instead, you must use a more complex and potentially inefficient query that has a nested \sh{SELECT} to compute the credit limit by folding events that set the initial credit and adjusting it. To make matters worse, a NoSQL-based event store will typically only support primary key-based lookup.
\end{quote}

Which highlight the benefits of implementing a pattern like ~\gls{ac:cqrs}, but also a big limitation of \gls{gl:es}, especially when the produced event log need to be utilized for auditing.

By implementing \gls{ac:cqrs} a separate query handler service is maintained. The query side keeps its data model synchronized with the command-side data model by subscribing to the events published by the command side and thus have direct access to the current transient state to efficiently handle such queries.

The following digram showcase how \gls{ac:cqrs} can be be implemented in a microservices system.

\includenamedimage[1]{figures/cqrs_microservices.png}{\gls{ac:cqrs} implementation in a microservices system \citep{richards2015software}}

\subsection{Eventual Consistency}

Eventual consistency forces developers to let go of guarantees that they would have in a system using current state and synchronous processing. In a \gls{ac:cqrs} system, an update sent through a command will not immediately be reflected in the result of a query. 

The system first needs to process the event into one or more projections~\citep{OVEREEM2021110970}, which leads to difficulties such as returning items to a client that in fact are already deleted. The reader systems are liable to be out of sync with the master (and each other) due to differences in timing with event propagation.

Getting people to understand eventual consistency is not easy. Eventual consistency forces developers to rethink the basic interactions of the user with the system. 

However, eventual consistency is a weaker form of consistency. The system guarantees that the query-side eventually will reflect the events produced in the command-side. However, there are no guarantees on how fast this will happen. 

A system with a large delay is unfeasible, because in that case queries will often return data that does not reflect the latest changes

\pagebreak

\section{Benefits}

\gls{gl:es} introduces a lot of challenges and complexity. As with any other pattern \gls{gl:es} is not always the solution. However a study conducted on 25 engineers with different roles and experiences reveled, that all systems under study benefit from \gls{gl:es}. Flexibility, debug-ability, reliability and auditability are common rational given for using \gls{gl:es}~\citep{OVEREEM2021110970}.

It is obvious, that the most repeated and almost always mentioned benefit of \gls{gl:es} is the events serialization into an audit log, that satisfies all characteristics as defined in chapter \ref{chap:adt}. Beside all other benefits, this makes \gls{gl:es} an optimized pattern to utilise when implementing \gls{gl:adt2} as discussed in section \ref{sec:adt2}

\gls{gl:es} also comes with debug-ability and customer support advantages. As put by~\citep{fowleres}:

\begin{quote}
  I chatted with someone who got their online accounts into an awkward state and phoned in for help. He was impressed that the helper was able to tell him exactly what he did and thus was able to figure out how to fix it.
\end{quote}

Providing such capability means exposing the audit trail to the support group so they can walk through a user's interaction. \citep{fowleres} acknowledge, that using \gls{gl:es} is not a requirement for such capabilities. Regular logging mechanisms are more than capable of achieving such results. However this assumes a logging infrastructure and utilisation that is capable of providing such information at ease, which is not always the case. More on this in chapter \ref{chap:sadt}

Furthermore, \gls{gl:es} allows developers and auditors to consider multiple time-lines (analogous to branching in version control systems) and recreate historic states or explore alternative histories by injecting hypothetical events when replaying. This means that even if the current tarnsaite state of the data has been corrupted or lost, it is still possible to recreate it from the event log. In contrast, traditional systems rely on the current state of the data. If this state is lost or corrupted, there is no way to recover the data's history without extensive backups.

Having the entire history of the state comes with the advantage of preserving the context, which allows for evidence based explanations of when and why something happened. Events can also be analysed for patterns in usage. Such information is impossible to extract from a store that only persists the latest state of the data.

As long as the stores criteria are meet a diverse range of databases can be used, such as relational, graph, or NoSql databases. The main goal of this store is to support the easy and fast retrieval of the data, in whatever form the system requires.

Another advantage of \gls{gl:es} is that it allows for easy implementation of fine-grained access controls. Because each event is stored as an individual record, it is possible to apply different access controls to different events, allowing for more precise control over who can access the data. This is in contrast to traditional systems, which typically apply access controls at the level of the entire database or table.

\gls{gl:es} can be a key element of a system, and that system can be as simple or as complex as the business domain requires it to be. It is useful to consider putting an event-sourced system in a part of the architecture that requires the preservation of context for all events, as this is where \gls{gl:es} is most effective. 