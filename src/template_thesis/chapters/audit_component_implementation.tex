% @author Hani Alshikh
%
\chapter{\gls{gl:ac}}\label{chap:ac}

As discussed in chapter \ref{chap:sadt} choosing the correct architecture for auditing is an important choice to make. Handling auditing and audit logging as first class citizens comes with its owen constraints and requirements.

In an industry where strict auditing requirements apply using patterns like \gls{gl:es} not only ensures a compliant audit log but also lays the ground work for broader auditing technologies like \gls{gl:adt2}

Combining \gls{ac:ddd}, \gls{gl:es}, and \gls{ac:cqrs} produces an architecture that is praised by many, when it comes to building audit first cloud systems.

To showcase the capabilities of such architecture and evaluate the development overhead compared to the gained benefits a real word example was chosen as a base to implement the \gls{gl:ac} from scratch.

The \gls{gl:ac} utilises the produced event log by the \gls{gl:es} implementation to offer an auditing \gls{ac:api}, that can be utilised by all kind of clients like auditing programs as described in section \ref{sec:adt2sa} or in this case by the \gls{gl:ab} implemented in chapter \ref{chap:ab}

\pagebreak

\section{Monoskope}\label{sec:m8}

Monoskope (short m8 spelled "mate") implements the management and operation of authenticating and authorizing entities in a \gls{gl:k8s} multi-cloud multi-cluster environments. It fulfills the needs of operators of the clusters as well as the needs of developers using the cloud infrastructure provided by the operators~\citep{monoskope}.

Monoskope was developed at \gls{gl:fc} with the primary use-case of managing developers access and permission requests to the \gls{gl:fc} cloud infrastructure. 

The \gls{gl:fc} cloud infrastructure spans multiple clusters, clouds and environments. Allowing compliant access, monitoring and logging authorisation as required by \citep{BaFinZAIT} is not an easy task.

The complex orchestration to allow for a coherent development experience powered by \gls{ac:k8s} regardless of the hosting provider meant dealing with multiple cluster and cloud instances with various setup requirements.

The cloud team at \gls{gl:fc} was faced with the challenge of satisfying all users requests while also ensuring a systematic and complaint trace of all authentication and authorization activities regardless of the backing setup complexity or compatibility.

Managing user roles and permissions while ensuring traceability and auditability required dealing with multiple \glspl{ac:api} on many different levels of complexity to eventually collect the produced logs and seilrize them into compliant audit logs using complex automation setups.

Ultimately this lead to the development of \gls{ac:m8} as an audit first system to ease and unify the different processes involved. The event-sourced architecture [\ref{sec:esa}] was chosen, as it satisfied all of the system requirements.

\pagebreak

\subsection{Architecture}

\Gls{ac:m8} was designed with auditing in mind. It uses the event-sourced architecture [\ref{sec:esa}] with a purpose built implementation of the \gls{gl:est} as well as the \gls{ac:cqrs} pattern.

\includenamedimage[1]{figures/monoskope-architecture}{Monoskope architecture ~\citep{monoskope}}
Since \gls{ac:m8} is meant to be used as the single source of truth authority for authenticating and authorizing users into different clusters on different cloud providers ensuring auditability and complying with different regulators requirements was not an option.

While the base system was implemented and the main features were complete \gls{ac:m8} still lacked a proper implementation to utilise the gained event log and offer an auditing \gls{ac:api}, that answers Auditors questions.

\Gls{ac:m8} has no classic \acrshort{ac:crud} database with tables for storing state, however, the transient state built based on that has. The following is the data model for the projected transient state:

\includenamedimage[1]{figures/monoskope_data_module}{Monoskope data module}

Generally users represent developers that are bound/a part of one or many teams/tenants that are bound/have access to one or many clusters.

\section{Requirements}

\Gls{ac:m8} has by nature a full audit log of every change to the system. This should be utilised to provide auditors and operators the ability to get detailed information of who is allowed to do what and why to answer questions like:

\begin{itemize}
  \item How did a user get a specific role?
  \item How did a user become a tenant member?
  \item What actions were taken by a user?
  \item etc\dots
\end{itemize}

Auditors have different backgrounds and technical knowledge thus all events musst have a human-readable representation.

It should be possible to utilise \gls{gl:es}'s temporal quires [\ref{sec:escp}] to get a users overview at any date and time 

\subsection{Use-Cases}

From the requirements overview the following use-cases were derived:

\mytable{acuc}{
  \def\arraystretch{1.2}
  \begin{tabularx}{\linewidth}{|l l X |}
    \hline
    \rowcolor{gray!20}
    \textbf{ID} & \textbf{Use-Case} & \textbf{Description}\\
    UC01 & Audit-Log for date-range & As an Auditor, I want to get all actions taken within a specific date-range\\
    UC02 & Audit-Log about a user & As an Auditor, I want to get all actions taken on a user\\
    UC03 & Audit-Log of user-actions & As an Auditor, I want to get all actions taken by a user\\
    UC04 & Audit-Log users overview & As an Auditor, I want to get an overview of all users at a specific timestamp, tenants they belongs to, and their roles within the system or tenants/clusters\\
    \hline
  \end{tabularx}
}{\gls{gl:ac} derived use-cases}

\subsection{Architectural Constraints}

\subsubsection{Technical Constraints}

\mytable{actc}{
  \begin{tabularx}{\linewidth}{|l l X |}
    \hline
    \rowcolor{gray!20}
    \textbf{ID} & \textbf{Constraint} & \textbf{Description}\\
    TC01 & Human-Readable representation & Event objects musst contain a human-readable representation \\
    TC02 & Programming language & \Gls{ac:m8} is written in \gls{gl:go}. No reason to use other languages \\
    TC03 & Middleware & \Gls{ac:m8} uses \gls{ac:grpc}. No reason to use or support any other framework \\
    \hline
  \end{tabularx}
}{\gls{gl:ac} technical constraints}

\subsubsection{Organisational Constraints}

\mytable{acoc}{
  \begin{tabularx}{\linewidth}{|l l X |}
    \hline
    \rowcolor{gray!20}
    \textbf{ID} & \textbf{Constraint} & \textbf{Description}\\
    OC01 & Deadline & Implementation musst be finalised before 31.01.2023\\
    OC02~\label{oc:ac02} & \gls{gl:12faktor} & Implementation musst adhere to the \gls{gl:12faktor} methodology\\
    \hline
  \end{tabularx}
}{\gls{gl:ac} organisational constraints}

\section{System Design}

\subsection{Scope and Context}

\subsubsection{Business Context}

\includenamedsvg{\includesvg}{diagrams/audit_component/audit_component_business_context.drawio}{\gls{gl:ac} business context diagram}

\subsubsection{Technical Context}

\includenamedsvg{\includesvg[width=\textwidth]}{diagrams/audit_component/audit_component_technical_context.drawio}{ \gls{gl:ac} technical context diagram}

\subsection{Solution Strategy}

\def\arraystretch{1.5}
\begin{xltabular}[H]{\linewidth}{|X l X X X |}
  \hline
  \rowcolor{gray!20}
  \textbf{Funktion} & \textbf{UCID} & \textbf{Semantics} & \textbf{Pre-condition} & \textbf{Post-condition}\\
  
  new\-Event\-Formatter\-Registry\-() & All & Creates a new EventFormatterRegistry & QueryHandler is initializing & EventFormatters can be registered \\

  register\-Event\-Formatter(\-eventFormatter,\-eventType\-) & All & Register EventFormatter for an event type & EventFormatter registry is initiated & EventFormatter can be used to format events of eventType \\

  new\-Audit\-Log\-Server\-(event\-Store\-Client, event\-Formatter\-Registry) & All & Creates server instance of the \gls{gl:ac} to handle client requests & \gls{gl:est}Client and EventFormatterRegistry are initialized & Audit-log server is ready to handle client requests  \\

  get\-By\-Date\-Range(date\-Range\-Request) & UC01 & Streams human-readable events within a date-range & Audit\-Log\-Server is running & Human-readable events were streamed to the client  \\

  new\-Human\-Readable\-Event(\-event\-) & All & creates a human-readable event of an event & EventFormatter for the corresponding event type was registered & A new event with human-readable details is created \\

  get\-By\-User(get\-By\-User\-Request) & UC02 & Streams formatted events caused by others actions on the given user & Audit\-Log\-Server is running & Human-readable events were streamed to the client  \\

  get\-Users\-Actions(get\-User\-Actions\-Request) & UC03 & Streams formatted events caused by the given user actions & Audit\-Log\-Server is running & Human-readable events were streamed to the client  \\

  get\-Users\-Overview(get\-Users\-Overview\-Request) & UC04 & Streams formatted events at the specified timestamp of users, tenants/clusters they belong to, and their roles & Audit\-Log\-Server is running & Human-readable events were streamed to the client  \\

  \hline
  \caption{\gls{gl:ac} solution strategy\label{tab:acst}}\\
\end{xltabular}

\subsection{Building Block View}

\subsubsection{Overall System White Box}

\includenamedsvg{\includesvg[pretex=\footnotesize]}{diagrams/audit_component/audit_component_component_diagram_white_box_level_1.drawio}{\gls{gl:ac} overall system component diagram}

\subparagraph{Level 1}

\header{Contained Building Blocks}

\subheader{\gls{gl:ac} Black Box}

\mytable{acsacoscbbwbs}{
  \begin{tabularx}{\linewidth}{|l X |}
    \hline
    \rowcolor{gray!20}
    \textbf{Component} & \textbf{Description}\\
    Audit & handles aggregating and formatting events for audit related queries\\
    \hline
  \end{tabularx}
}{\gls{gl:ac} contained building blocks component black box}

\mytable{acsacdoscbbwbs}{
  \begin{tabularx}{\linewidth}{|l X |}
    \hline
    \rowcolor{gray!20}
    \textbf{Interface} & \textbf{Description}\\
    AuditLogClient & handles communication with the audit log server\\
    \hline
  \end{tabularx}
}{\gls{gl:ac} contained building blocks interface black box}

\subparagraph{Level 2}

\header{\gls{gl:ac} White Box}

\includenamedsvg{\includesvg[width=\textwidth,pretex=\relscale{0.2}]}{diagrams/audit_component/audit_component_component_diagram_white_box_level_2.drawio}{\gls{gl:ac} class diagram}

\mytable{accbbwb}{
  \begin{tabularx}{\linewidth}{|l X |}
    \hline
    \rowcolor{gray!20}
    \textbf{Object} & \textbf{Description}\\
    CertificateEventFormatter & Formates certificate events in human-readable format \\
    ClusterEventFormatter & Formates cluster events in human-readable format \\
    TenantEventFormatter & Formates tenant events in human-readable format \\
    UserEventFormatter & Formates user events in human-readable format \\
    BaseEventFormatter & Base implementation of the EventFormatter interface to generalize common methods \\
    EventFormatterRegistry & Register EventFormatter for event type \\
    AuditFormatter & Creates a human-readable event of a given event \\
    AuditLogServer & Handles communication with the AuditLogClient \\
    \hline
  \end{tabularx}
}{\gls{gl:ac} class diagram}

\subsection{Runtime View}

\subsubsection{UC01-04: Audit-Log *}

\includenamedsvg{\includesvg[width=\textwidth,pretex=\relsize{-10}]}{diagrams/audit_component/audit_component_uc01-04_get_audit-log_sequence_diagram.drawio}{\gls{gl:ac} \hyperref[tab:abuc]{UC01-04} sequence diagram}