% @author Hani Alshikh
%
\chapter{Audit Component}\label{chap:ac}

Wenn mehrere Teil-Kapitel zu strukturieren sind: Schreiben Sie zu jedem Teil-Kapitel eine Ein- leitung ("Hier wird die folgende Fragestellung untersucht...") und eine Ausleitung ("Hiermit ist erreicht: ... Die folgenden Probleme sind aber noch offen:...").

The event sourced system \citep{OVEREEM2021110970}

Auditing Framework \citep{5427384}
Event logs and process mining techniques enable new forms of auditing. Rather than sampling a small set of cases, auditors can consider the whole process and all of its instances. Moreover, they can do this continuously.

Figure 1 shows an Auditing 2.0 framework based on process mining. “Current data” events are cases that are still running, while “Historic data” events are completed cases. The figure also shows two types of process models: De jure models describe a desired or required way of working, while de facto models aim to describe reality with potential violations of the boundaries defined in de jure models (W. M. P. van der Aalst et al., “Conceptual Model for On Line Auditing,” tech. report BPM-09-19 BPMcenter.org, 2009).

\section{Monoskope}\label{sec:m8}

\gls{ac:m8} implements the management and operation of tenants, users and their roles in a Kubernetes multi-cluster environment. It fulfills the needs of operators of the clusters as well as the needs of developers using the cloud infrastructure provided by the operators.~\todo{rewrite m8 overview}

\gls{ac:m8} is an event-sourced system...~\todo{TODO}

\section{Requirements}

\todo{clreaify how this should be  written? meeting, disscussion parties, etc...}

\gls{ac:m8} has by nature a full audit log of every change to the system.This should be utilised to provide auditors and operators the ability to get detailed information of who is allowed to do what and why to answer questions like:

\begin{itemize}
  \item How did a user get a specific role?
  \item How did a user become a tenant member?
  \item What actions were taken by a user?
  \item \dots
\end{itemize}

Auditors have different backgrounds and technical knowledge thus all events musst have a human-readable representation.

It should be possible to utilise \gls{gl:es}'s temporal quires \todo{add chapter refrence} to get a system overview at a specific date and time 

...~\citep{SAIP}

\subsection{Use-Cases}

From the requirements overview the following use-cases are derived

\begin{table}[H]
  \begin{center}
    \caption{Audit component derived use-cases}
    \label{tab:acuc}
    \def\arraystretch{1.5}
    \begin{tabularx}{\linewidth}{|l l X |}
      \hline
      \rowcolor{gray!20}
      \textbf{ID} & \textbf{Use-Case} & \textbf{Description}\\
      UC01 & get audit-log for date-range & As an auditor, I want to get all actions taken within a specific date-range\\
      UC02 & get audit-log about a user & As an auditor, I want to get all actions taken on a user\\
      UC03 & get audit-log of user-actions & As an auditor, I want to get all actions taken by a user\\
      UC04 & get audit-log users overview & As an auditor, I want to get an overview of all users at a specific timestamp, tenants they belongs to, and their roles within the system or tenants/clusters\\
      \hline
    \end{tabularx}
  \end{center}
\end{table}

% \subsection{Stakeholders}

\subsection{Architectural Constraints}

\subsubsection{Technical Constraints}

\begin{table}[H]
  \begin{center}
    \caption{Audit component technical constraints}
    \label{tab:actc}
    \def\arraystretch{1.5}
    \begin{tabularx}{\linewidth}{|l l X |}
      \hline
      \rowcolor{gray!20}
      \textbf{ID} & \textbf{Constraint} & \textbf{Description}\\
      TC01 & human-readable representation & log structs musst contain a human-readable representation\\
      TC02 & Programming language & \gls{ac:m8} is written in \gls{gl:go}. No reason to use other language\\
      TC03 & Middleware & \gls{ac:m8} uses \gls{ac:grpc}. No reason to use or support any other framework\\
      \hline
    \end{tabularx}
  \end{center}
\end{table}

\subsubsection{Organisational Constraints}

\begin{table}[H]
  \begin{center}
    \caption{Audit component organisational constraints}
    \label{tab:acoc}
    \def\arraystretch{1.5}
    \begin{tabularx}{\linewidth}{|l l X |}
      \hline
      \rowcolor{gray!20}
      \textbf{ID} & \textbf{Constraint} & \textbf{Description}\\
      OC01 & Deadline & implementation musst be finalised before 31.01.2023\\
      OC02~\label{oc:ac02} & \gls{gl:12faktor} & adhere to the \gls{gl:12faktor} methodology\\
      \hline
    \end{tabularx}
  \end{center}
\end{table}

\section{System Design}

% \subsection{Enabling Technologies}

\subsection{Scope and Context}

\subsubsection{Business Context}

\includenamedsvg{\includesvg}{diagrams/audit_component/audit_component_business_context.drawio}{Audit Component Business Context Diagram}

\subsubsection{Technical Context}

\includenamedsvg{\includesvg[width=\textwidth]}{diagrams/audit_component/audit_component_technical_context.drawio}{ Audit Component Technical Context Diagram}

\subsection{Solution Strategy}

\begin{table}[H]
  \footnotesize
  \begin{center}
    \caption{Audit component solution strategy}
    \label{tab:acss}
    \def\arraystretch{1.5}
    \begin{tabularx}{\linewidth}{|l X l X X X |}
      \hline
      \rowcolor{gray!20}
      \textbf{Actor} & \textbf{Funktion} & \textbf{UCID} & \textbf{Semantics} & \textbf{Pre-condition} & \textbf{Post-condition}\\
      QueryHandler & New\-Audit\-Log\-Server\-(Event\-Store\-Client, Event\-Formatter\-Registry) & All & Creates server instance of the audit log component to handle grpc client requests & \gls{gl:ef} server is running and \glspl{gl:ef} are registered & Audit-log server is ready to handle client requests  \\
      AuditLogClient & Get\-By\-Date\-Range(Range\-Request, Stream) & UC01 & streams formatted events & Audit\-Log\-Server is running & human-readable events were streamed to the client  \\
      ... & ... & ... & ... & ... & ...  \\
      \hline
    \end{tabularx}
  \end{center}
\end{table}
...\todo{finish the table}

\subsection{Building Block View}

\subsubsection{Overall System White Box}

\includenamedsvg{\includesvg[pretex=\footnotesize]}{diagrams/audit_component/audit_component_component_diagram_white_box_level_1.drawio}{Overall System Component Diagram}

\subparagraph{Level 1}

\header{Contained Building Blocks}

\begin{table}[H]
  \begin{center}
    \caption{Audit component overall system contained building blocks white box}
    \label{tab:acoscbbwb}
    \def\arraystretch{1.5}
    \begin{tabularx}{\linewidth}{|l X |}
      \hline
      \rowcolor{gray!20}
      \textbf{Component} & \textbf{Description}\\
      Audit & handles aggregating and formatting events for audit related queries\\
      \hline
    \end{tabularx}
  \end{center}
\end{table}

\header{Audit Component Black Box}

\begin{table}[H]
  \begin{center}
    \caption{Audit component black box}
    \label{tab:acbb}
    \def\arraystretch{1.5}
    \begin{tabularx}{\linewidth}{|l X |}
      \hline
      \rowcolor{gray!20}
      \textbf{Interface} & \textbf{Description}\\
      AuditLogClient & handles communication with the audit log server\\
      \hline
    \end{tabularx}
  \end{center}
\end{table}

\subparagraph{Level 2}

\header{Audit Component White Box}

% \caption{Audit Component White Box}
\includenamedsvg{\includesvg[width=\textwidth,pretex=\footnotesize]}{diagrams/audit_component/audit_component_component_diagram_white_box_level_2.drawio}{Audit Component Diagram}

\begin{table}[H]
  \begin{center}
    \caption{Audit component contained building blocks white box}
    \label{tab:accbbwb}
    \def\arraystretch{1.5}
    \begin{tabularx}{\linewidth}{|l X |}
      \hline
      \rowcolor{gray!20}
      \textbf{Component} & \textbf{Description}\\
      Audit & handles audit client queries\\
      EventFormatterRegistry & register and initialise \glspl{gl:ef} based on the event type\\
      EventFormatter & formats an event in human-readable format\\
      \hline
    \end{tabularx}
  \end{center}
\end{table}

\subparagraph{Event Component Black Box}.

\begin{table}[H]
  \begin{center}
    \caption{Event component black box}
    \label{tab:ecbb}
    \def\arraystretch{1.5}
    \begin{tabularx}{\linewidth}{|l X |}
      \hline
      \rowcolor{gray!20}
      \textbf{Interface} & \textbf{Description}\\
      EventFormatterRegistry & register and initialise \glspl{gl:ef} based on the event type\\
      EventFormatter & formats an event in human-readable format\\
      \hline
    \end{tabularx}
  \end{center}
\end{table}

\subparagraph{Level 3}

\todo{Class Digram}

\subsection{Runtime View}

\subsection{Deployment View}

\section{Design Decisions}

\subsection{DD01: Registry pattern}

\section{Technical Decisions}

\subsection{TD01: Limit get user actions query to max 1 year}

a general restructure for the event structure will happen sooner or later (which should allow for a much better implementation)
for now, limiting the query to a max 1 year should be enough as a preventive measure until the restructuring to filter on the store level is implemented
All queries are bound to a time-out which will stop the query if it takes too long In case there are too many events within the limiting period.

