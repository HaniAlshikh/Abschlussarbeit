% @author Hani Alshikh
%
\chapter{Audit Component}

Wenn mehrere Teil-Kapitel zu strukturieren sind: Schreiben Sie zu jedem Teil-Kapitel eine Ein- leitung ("Hier wird die folgende Fragestellung untersucht...") und eine Ausleitung ("Hiermit ist erreicht: ... Die folgenden Probleme sind aber noch offen:...").

The event sourced system \cite{OVEREEM2021110970}

Auditing Framework \cite{5427384}
Event logs and process mining techniques enable new forms of auditing. Rather than sampling a small set of cases, auditors can consider the whole process and all of its instances. Moreover, they can do this continuously.

Figure 1 shows an Auditing 2.0 framework based on process mining. “Current data” events are cases that are still running, while “Historic data” events are completed cases. The figure also shows two types of process models: De jure models describe a desired or required way of working, while de facto models aim to describe reality with potential violations of the boundaries defined in de jure models (W. M. P. van der Aalst et al., “Conceptual Model for On Line Auditing,” tech. report BPM-09-19 BPMcenter.org, 2009).

\section{Monoskope}

\gls{ac:m8} implements the management and operation of tenants, users and their roles in a Kubernetes multi-cluster environment. It fulfills the needs of operators of the clusters as well as the needs of developers using the cloud infrastructure provided by the operators.~\todo{rewrite m8 overview}

\gls{ac:m8} is an event-sourced system...~\todo{TODO}



\section{Requirements}

\todo{clreaify how this should be  written? meeting, disscussion parties, etc...}

\gls{ac:m8} has by nature a full audit log of every change to the system.This should be utilised to provide auditors and operators the ability to get detailed information of who is allowed to do what and why to answer questions like:

\begin{itemize}
  \item How did a user get a specific role?
  \item How did a user become a tenant member?
  \item What actions were taken by a user?
  \item \dots
\end{itemize}

Auditors have different backgrounds and technical knowledge thus all events musst have a human-readable representation.

It should be possible to utilise \gls{ac:es}'s temporal quires \todo{add chapter refrence} to get a system overview at a specific date and time 

...~\citep{SAIP}

\subsection{Use-Cases}

\subsection{Stakeholders}

\subsection{Architectural Constraints}


\section{System Design}

\subsection{Enabling Technologies}

\subsection{Scope and Context}

\subsection{Solution Strategy}

\subsection{Building Block View}

\subsection{Runtime View}

\subsection{Deployment View}


\section{Design Decisions}

\subsection{DD01: Registry pattern}


\section{Technical Decisions}

